% Created 2021-06-03 Thu 14:40
% Intended LaTeX compiler: pdflatex
\documentclass[11pt]{article}
\usepackage[utf8]{inputenc}
\usepackage[T1]{fontenc}
\usepackage{graphicx}
\usepackage{grffile}
\usepackage{longtable}
\usepackage{wrapfig}
\usepackage{rotating}
\usepackage[normalem]{ulem}
\usepackage{amsmath}
\usepackage{textcomp}
\usepackage{amssymb}
\usepackage{capt-of}
\usepackage{hyperref}
\author{Noboru Murata}
\date{\today}
\title{}
\hypersetup{
 pdfauthor={Noboru Murata},
 pdftitle={},
 pdfkeywords={},
 pdfsubject={},
 pdfcreator={Emacs 27.2 (Org mode 9.3.7)}, 
 pdflang={English}}
\begin{document}

\tableofcontents


\section{概要}
\label{sec:orga0525e3}
本講義では
フーリエ変換に代表される基底による信号の表現と取り扱い,
簡単なフィルタの理論を学ぶことを目的とします.

\subsection{講義資料}
\label{sec:org12584aa}
\begin{itemize}
\item \href{https://noboru-murata.github.io/signal-processing/pdfs/signal-processing.pdf}{信号処理}
\end{itemize}
随時修正します.

\subsection{参考資料}
\label{sec:org1e3cd38}
必要な参考書については講義中に指示します.

\section{講義1}
\label{sec:orge635631}
\textit{<2021-04-30 Fri> } 更新
\subsection{スライド}
\label{sec:org0621a66}
\{\{< myslide base="signal-processing" name="slide01" >\}\}
\subsection{ハンドアウト}
\label{sec:orge60b084}
\begin{itemize}
\item \href{https://noboru-murata.github.io/signal-processing/pdfs/slide01.pdf}{PDF file}
\end{itemize}

\section{講義2}
\label{sec:orgce3bbc6}
\textit{<2021-04-30 Fri> } 更新
\subsection{スライド}
\label{sec:orgfd98243}
\{\{< myslide base="signal-processing" name="slide02" >\}\}
\subsection{ハンドアウト}
\label{sec:org74d3d28}
\begin{itemize}
\item \href{https://noboru-murata.github.io/signal-processing/pdfs/slide02.pdf}{PDF file}
\end{itemize}

\section{講義3}
\label{sec:org708b25f}
\textit{<2021-04-30 Fri> } 更新
\subsection{スライド}
\label{sec:org71a467d}
\{\{< myslide base="signal-processing" name="slide03" >\}\}
\subsection{ハンドアウト}
\label{sec:org6fcea20}
\begin{itemize}
\item \href{https://noboru-murata.github.io/signal-processing/pdfs/slide03.pdf}{PDF file}
\end{itemize}

\section{講義4}
\label{sec:orgb7cdfa3}
\textit{<2021-04-30 Fri> } 更新
\subsection{スライド}
\label{sec:org437975e}
\{\{< myslide base="signal-processing" name="slide04" >\}\}
\subsection{ハンドアウト}
\label{sec:org0672e59}
\begin{itemize}
\item \href{https://noboru-murata.github.io/signal-processing/pdfs/slide04.pdf}{PDF file}
\end{itemize}

\section{講義5}
\label{sec:org38a4656}
\textit{<2021-05-07 Fri> } 更新
\subsection{スライド}
\label{sec:org98672d1}
\{\{< myslide base="signal-processing" name="slide05" >\}\}
\subsection{ハンドアウト}
\label{sec:org3180d15}
\begin{itemize}
\item \href{https://noboru-murata.github.io/signal-processing/pdfs/slide05.pdf}{PDF file}
\end{itemize}

\section{講義6}
\label{sec:orgbbc48bc}
\textit{<2021-05-14 Fri> } 更新
\subsection{スライド}
\label{sec:orge28c80b}
\{\{< myslide base="signal-processing" name="slide06" >\}\}
\subsection{ハンドアウト}
\label{sec:orgad04f31}
\begin{itemize}
\item \href{https://noboru-murata.github.io/signal-processing/pdfs/slide06.pdf}{PDF file}
\end{itemize}

\section{講義7}
\label{sec:orgadc47a3}
理解度の確認
\section{講義8}
\label{sec:org7679cba}
\textit{<2021-06-02 Wed> } 更新
\subsection{スライド}
\label{sec:org3abdb30}
\{\{< myslide base="signal-processing" name="slide08" >\}\}
\subsection{ハンドアウト}
\label{sec:org59b5896}
\begin{itemize}
\item \href{https://noboru-murata.github.io/signal-processing/pdfs/slide08.pdf}{PDF file}
\end{itemize}

\section{講義9}
\label{sec:org04e2b0d}
\textit{<2021-06-02 Wed> } 更新
\subsection{スライド}
\label{sec:org9e6c635}
\{\{< myslide base="signal-processing" name="slide09" >\}\}
\subsection{ハンドアウト}
\label{sec:org899d6cc}
\begin{itemize}
\item \href{https://noboru-murata.github.io/signal-processing/pdfs/slide09.pdf}{PDF file}
\end{itemize}

\section{講義10}
\label{sec:orgad1e92f}
\textit{<2021-06-02 Wed> } 更新
\subsection{スライド}
\label{sec:orga77bdc4}
\{\{< myslide base="signal-processing" name="slide10" >\}\}
\subsection{ハンドアウト}
\label{sec:org4bb577f}
\begin{itemize}
\item \href{https://noboru-murata.github.io/signal-processing/pdfs/slide10.pdf}{PDF file}
\end{itemize}

\section{講義11}
\label{sec:org5a8589c}
理解度の確認
\section{講義12}
\label{sec:org7de66d5}
\textit{<2021-06-03 Thu> } 更新
\subsection{スライド}
\label{sec:orgcd1139d}
\{\{< myslide base="signal-processing" name="slide12" >\}\}
\subsection{ハンドアウト}
\label{sec:org7c2d168}
\begin{itemize}
\item \href{https://noboru-murata.github.io/signal-processing/pdfs/slide12.pdf}{PDF file}
\end{itemize}

\section{講義13}
\label{sec:org0269dc7}
\textit{<2021-06-02 Wed> } 更新
\subsection{スライド}
\label{sec:orgec1163c}
\{\{< myslide base="signal-processing" name="slide13" >\}\}
\subsection{ハンドアウト}
\label{sec:orgb841726}
\begin{itemize}
\item \href{https://noboru-murata.github.io/signal-processing/pdfs/slide13.pdf}{PDF file}
\end{itemize}

\section{講義14}
\label{sec:org22a2494}
\textit{<2021-06-03 Thu> } 更新
\subsection{スライド}
\label{sec:org29dfc33}
\{\{< myslide base="signal-processing" name="slide14" >\}\}
\subsection{ハンドアウト}
\label{sec:org430154b}
\begin{itemize}
\item \href{https://noboru-murata.github.io/signal-processing/pdfs/slide14.pdf}{PDF file}
\end{itemize}

\section{講義15}
\label{sec:org54c3548}
理解度の確認

\section{講義の進め方}
\label{sec:orgee3eceb}
\subsection{講義ノート}
\label{sec:orgddffaa8}
Moodle に URL を掲載しました.

\subsection{過去の試験問題}
\label{sec:org995d9e2}
Moodle に3年分を掲載しました.

\section{スライドの使い方}
\label{sec:orgdcbf98c}
スライドは
\href{https://revealjs.com}{reveal.js}
を使って作っています.

スライドを click して "?" を入力すると
shortcut key が表示されますが,
これ以外にも以下の key などが使えます.

\subsection{フルスクリーン}
\label{sec:org911a883}
\begin{itemize}
\item f フルスクリーン表示
\item esc 元に戻る
\end{itemize}
\subsection{黒板}
\label{sec:orgde7fdc5}
\begin{itemize}
\item w スライドと黒板の切り替え (toggle)
\item x/y チョークの色の切り替え (巡回)
\item c 消去
\end{itemize}
\subsection{メモ書き}
\label{sec:org2fa8275}
\begin{itemize}
\item e 編集モードの切り替え (toggle)
\item x/y ペンの色の切り替え (巡回)
\item c 消去
\end{itemize}
\end{document}